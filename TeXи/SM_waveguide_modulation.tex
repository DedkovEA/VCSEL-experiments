% \documentclass[12pt,a4paper]{article}
\documentclass[12pt, notitlepage]{report}
\usepackage[utf8]{inputenc}
\usepackage[T1]{fontenc}


\usepackage[left=1in, right=1in, top=1in, bottom=1in]{geometry}

\usepackage{titling}
%\usepackage{lipsum}

\usepackage{amsmath}
\usepackage{amsfonts}
\usepackage{amssymb}
\usepackage{graphicx}
\usepackage{float}
%\usepackage[margin=2cm]{geometry}
\usepackage{caption}
\usepackage{dsfont}

\pretitle{\begin{center}\Huge\bfseries}
	\posttitle{\par\end{center}\vskip 0.5em}
\preauthor{\begin{center}\Large\ttfamily}
	\postauthor{\end{center}}
\predate{\par\large\centering}
\postdate{\par}

\title{SM optical fiber}
\author{Evgeniy Dedkov}
\date{\today}
\begin{document}
	
	\newcommand\scalemath[2]{\scalebox{#1}{\mbox{\ensuremath{\displaystyle #2}}}}
	\newcommand{\mcm}{\mathcal{M}}
	
	\maketitle
	\thispagestyle{empty}
	
	\begin{abstract}
		This article consider modulation of SM optic fiber via using differential equations and RK-4 method.
	\end{abstract}
	
	\section{Jones formalism}
	Consider absolutely polarized EM wave with fixed circular frequency $\omega$. Lets fix laboratory coordinate system and treat all optical elements as positioned on the straight line. Then we need to describe only 2 amplitudes of EM field: $E_x$ and $E_y$. They are complex scalars, changing while light passes through optical elements and medium and form a Jones vector:
	$$
	\begin{pmatrix}
		E_x \\
		E_y
	\end{pmatrix}
	$$
	
	Starting from $
	\begin{pmatrix}
		E_{x}^0 \\
		E_{y}^0
	\end{pmatrix}
	$ light passes through different optical elements and medium, which can be described as multiplying Jones vector on Jones matrices of those elements:
	$$
		\begin{pmatrix}
		E_{x}^\text{end} \\
		E_y^\text{end}
	\end{pmatrix} = \mathcal{M}_N\dots\mathcal{M}_2\mathcal{M}_1\begin{pmatrix}
	E_{x}^0 \\
	E_{y}^0
\end{pmatrix}
	$$
	\section{Equation on Jones matrix of SM fiber}
	Consider single mode optical fiber of length $L$. Let $\mathcal{M}(z)$ be Jones matrix of fiber piece starting from 0 to $z$, $\theta(z)$ - angle between x-axis of lab system and axis of x-mode in fiber, $n_x(z)$, $n_y(z)$~--- refractive indices at coordinate $z$. 
	
	Now we will look at small part of fiber of length $dz$, starting from $z$. For Jones matrix one can then write:
	$$
	\mcm(z+dz) = \mathcal{R}^{-1}[\theta(z)]\begin{pmatrix}
		e^{i\cdot d\phi_x(z)} & 0 \\
		0 & e^{i\cdot d\phi_y(z)}
	\end{pmatrix}\mathcal{R}[\theta(z)]\mcm(z)
	$$ 
	where $d\phi_{x,y}(z) = \frac{\omega n_{x,y}(z)}{c}dz$ and $\mathcal{R}[\theta] = \begin{pmatrix}
		\cos\theta & \sin\theta\\ -\sin\theta & \cos\theta
	\end{pmatrix}$~--- rotation matrix.

	Lets subtract $\mcm(z)$ from both parts of equation. After some simplification we will get:
	$$
	d\mcm(z) = \begin{pmatrix}
		-1 + e^{i\cdot d\phi_x(z)} \cos^2\theta(z) + e^{i\cdot d\phi_y(z)} \sin^2\theta(z) & (e^{i\cdot d\phi_x(z)} - e^{i\cdot d\phi_y(z)}) \sin\theta(z)\cos\theta(z) \\
		(e^{i\cdot d\phi_x(z)} - e^{i\cdot d\phi_y(z)}) \sin\theta(z)\cos\theta(z) & -1 + e^{i\cdot d\phi_x(z)} \sin^2\theta(z) + e^{i\cdot d\phi_y(z)} \cos^2\theta(z) 
	\end{pmatrix}\mcm(z)
	$$
	here we introduced $d\mcm(z) = \mcm(z+dz) - \mcm(z)$.
	
	Leaving only first order of $d\phi_{x,y}$ and dividing on $dz$, we obtain differential equation on Jones matrix of SM fiber:
	\begin{multline*}
		\dfrac{d\mcm(z)}{dz} = i\left\{\dfrac{d\phi_x}{dz}\begin{pmatrix}
		\cos^2\theta(z) & \sin\theta(z)\cos\theta(z) \\
		\sin\theta(z)\cos\theta(z) & \sin^2\theta(z)
	\end{pmatrix} + \right. \\ \left.
 		\dfrac{d\phi_y(z)}{dz}\begin{pmatrix}
	\sin^2\theta(z) & -\sin\theta(z)\cos\theta(z) \\
	-\sin\theta(z)\cos\theta(z) & \cos^2\theta(z)
	\end{pmatrix}
	\right\}\mcm(z)
	\end{multline*}
	with obvious initial condition: $\mcm(0)=\mathds{1}$. Substituting $d\phi$ explicitly we finally obtain:
	\begin{multline}
		\label{main}
		\dfrac{d\mcm(z)}{dz} = i\dfrac{\omega}{c}\left\{n_x(z)\begin{pmatrix}
			\cos^2\theta(z) & \sin\theta(z)\cos\theta(z) \\
			\sin\theta(z)\cos\theta(z) & \sin^2\theta(z)
		\end{pmatrix} + \right. \\ \left.
		n_y(z)\begin{pmatrix}
			\sin^2\theta(z) & -\sin\theta(z)\cos\theta(z) \\
			-\sin\theta(z)\cos\theta(z) & \cos^2\theta(z)
		\end{pmatrix}
		\right\}\mcm(z)
	\end{multline}

	Note, that we can freely add terms, proportional to $i\mcm(z)$ to the right side of eq.~\eqref{main} if there is no need to keep track of phase. The reason for it is:
	\begin{equation*}
		\dfrac{d\left(e^{i\alpha(z)}\mcm(z)\right)}{dz} = e^{i\alpha(z)}\left(\dfrac{d\mcm(z)}{dz} + i\dfrac{\alpha(z)}{dz}\mcm(z)\right)
	\end{equation*}
	so, solving~\eqref{main} with an arbitrary additional term $if(z)\mcm(z)$ will result in phase-shifted solution:
	\begin{equation*}
		\mcm'(z) = e^{i\alpha(z)}\mcm(z), \qquad \alpha(z) = \int\limits_0^zf(t)dt
	\end{equation*}

	However, if we consider addition of term $-i \dfrac{\omega}{c} n_x(z) \mcm(z)$ we will obtain much simpler equation:
	\begin{equation}
		\label{main2}
		\dfrac{d\mcm'(z)}{dz} = i\dfrac{\omega}{c}\Delta n(z)\begin{pmatrix}
			\sin^2\theta(z) & -\sin\theta(z)\cos\theta(z) \\
			-\sin\theta(z)\cos\theta(z) & \cos^2\theta(z)
		\end{pmatrix}\mcm'(z)
	\end{equation}
	where $\Delta n(z) = n_y(z) - n_x(z)$. And the true Jones matrix will be:
	\begin{equation}
		\label{true}
		\mcm(z) = e^{i\int\limits_0^z \frac{\omega}{c} n_x(t) dt}\mcm'(z)
	\end{equation}

\section{Jones matrix of SM fiber properties}
From now lets pay no attention on light losses. It is reasonable in small optical systems. In this case one can easily show unitarity of $\mcm$, which should simply be the result of losses absence. Obviously, $\mcm(0)^\dagger\mcm(0) = \mathds{1}$, and:
\begin{equation*}
	\dfrac{d\left(\mcm^\dagger(z)\mcm(z)\right)}{dz} = \dfrac{d\mcm^\dagger(z)}{dz}\mcm(z) + \mcm^\dagger(z)\dfrac{d\mcm(z)}{dz} = \mcm^\dagger(z) A^\dagger(z) \mcm(z) + \mcm^\dagger(z)A(z)\mcm(z) = 0
\end{equation*}
since from~\eqref{main} we see $A^\dagger(z)=-A(z)$ if refractive indices $n_{x,y}(z)$ is real. 

Furthermore, $\mcm(z)$ belongs to $SU(2)$ group, since it starts in $\mcm(0)=\mathds{1}$ and smoothly evolve with change of $z$. It is well known that such matricies can be described with only 2 complex arguments $\zeta, \xi$, such that $|\zeta|^2+|\xi|^2=1$, so in our case:
\begin{equation*}
	\mcm(z) = \begin{pmatrix}
		\zeta(z) & -\bar{\xi}(z) \\
		\xi(z) &  \bar{\zeta}(z)
	\end{pmatrix}, \qquad |\zeta(z)|^2+|\xi(z)|^2=1
\end{equation*}

One key feature about such representation is that matrix differential equation~\eqref{main} is now reduced to single vector equation:
\begin{equation}
	\label{mainvec}
	\dfrac{d}{dz} \begin{pmatrix}
		\zeta(z) \\
		\xi(z)
	\end{pmatrix} = A(z) \begin{pmatrix}
	\zeta(z) \\
	\xi(z)
\end{pmatrix}
\end{equation}
where $A(z)$ is the following:
\begin{multline*}
	A(z) = i\dfrac{\omega}{c}\left\{n_x(z)\begin{pmatrix}
		\cos^2\theta(z) & \sin\theta(z)\cos\theta(z) \\
		\sin\theta(z)\cos\theta(z) & \sin^2\theta(z)
	\end{pmatrix} + \right. \\ \left.
	n_y(z)\begin{pmatrix}
		\sin^2\theta(z) & -\sin\theta(z)\cos\theta(z) \\
		-\sin\theta(z)\cos\theta(z) & \cos^2\theta(z)
	\end{pmatrix}
	\right\}
\end{multline*}


\section{Statistical properties of $\mcm$}
Consider a set of arbitrary SM optical fibers of length $L$ with $n_{x,y}(z)$ distributed independently in each point $z$ and having mean values $\mathbb{E}n_x = \bar{n}_x$, $\mathbb{E}n_y = \bar{n}_y$. Let $\theta(z)$ be distributed uniformly and also independently in each point. As we will see, in such case mean Jones matrix of such set of fibers can be derived analytically.

First of all, we will need to perform a step-back, and rewrite $\mcm(z+dz)$ in terms of matrix product:
\begin{multline*}
	\mcm(z+ dz) = \left\{\mathds{1} + i\dfrac{\omega}{c}\left[n_x(z)
	\begin{pmatrix}
		\cos^2\theta(z) & \sin\theta(z)\cos\theta(z) \\
		\sin\theta(z)\cos\theta(z) & \sin^2\theta(z)
	\end{pmatrix} + \right.\right.\\ \left.\left.
	n_y(z) 
	\begin{pmatrix}
		\sin^2\theta(z) & -\sin\theta(z)\cos\theta(z) \\
		-\sin\theta(z)\cos\theta(z) & \cos^2\theta(z)
	\end{pmatrix}
	\right]dz\right\}\mcm(z) = B(z,dz)\mcm(z)
\end{multline*}

The solution for $\mcm(L)$ thus will be:
\begin{equation}
	\label{M}
	\mcm(L) = \lim\limits_{N\rightarrow\infty}\prod\limits_{j=0}^{N-1} B\left(\dfrac{L}{N}j, \dfrac{L}{N}\right)
\end{equation}
and for average fiber Jones matrix we may write\footnote{With respect to Lebesgue theorem of dominated convergence, because all $\mcm$ matrices are of $SU(2)$ group, so it is always true, that $|\mcm_{\alpha\beta}| \leq 1$, $|B(\dots)_{\alpha\beta}| \leq 1$}:
\begin{equation*}
	\mathbb{E}\mcm(L) = \lim\limits_{N\rightarrow\infty}\mathbb{E}\prod\limits_{j=0}^{N-1} B\left(\dfrac{L}{N}j, \dfrac{L}{N}\right)
\end{equation*}

Notice, that for independent random matrices $M^1, M^2$ it will be fulfilled:
\begin{equation*}
	\mathbb{E}\left(M^1 M^2\right) = \begin{Vmatrix}
		\mathbb{E}\left(\sum_\gamma M^1_{\alpha\gamma}M^2_{\gamma\beta}\right) 
	\end{Vmatrix} = \begin{Vmatrix}
	\sum_\gamma \mathbb{E}M^1_{\alpha\gamma} \mathbb{E}M^2_{\gamma\beta}
\end{Vmatrix} = \mathbb{E}M^1\cdot \mathbb{E}M^2
\end{equation*}


Since all parameters of $B$ matrices in different points $z = \dfrac{L}{N}j$ are independent, then we may write:
\begin{equation*}
	\mathbb{E}\mcm(L) = \lim\limits_{N\rightarrow\infty}\prod\limits_{j=0}^{N-1} \mathbb{E}\left[B\left(\dfrac{L}{N}j, \dfrac{L}{N}\right)\right] = \lim\limits_{N\rightarrow\infty} \left(\mathbb{E}[B]\left(\dfrac{L}{N}\right)\right)^N
\end{equation*}

Average $B\left(\dfrac{L}{N}j, \dfrac{L}{N}\right)$ can be calculated with respect to all assumptions above. We have:
\begin{equation*}
	\mathbb{E}[B]\left(\dfrac{L}{N}\right) = \begin{pmatrix}
		1 + i\dfrac{\omega}{2c}\left(\bar{n}_x + \bar{n}_y\right)\dfrac{L}{N} & 0 \\
		0 & 1 + i\dfrac{\omega}{2c}\left(\bar{n}_x + \bar{n}_y\right)\dfrac{L}{N}
	\end{pmatrix}
\end{equation*}
from which its simple to find:
\begin{equation}
	\mathbb{E}\mcm(L) = \mathds{1}\lim\limits_{N\rightarrow\infty}\left(1 + i\dfrac{\omega}{2c}\left(\bar{n}_x + \bar{n}_y\right)\dfrac{L}{N} \right) =  e^{i\frac{\omega L}{c}\frac{\bar{n}_x + \bar{n}_y}{2}}\mathds{1}
\end{equation}

\label{problem}
This result means, that there is no point in modeling average SM fiber with independently distributed $\theta$, $n_x$ and $n_y$, because the result will be obvious. 

One may argue, that finite length of fiber pieces will change this result somehow, but all the changes occurs in $B(z,dz)$ matrix, and after averaging with respect to uniformly and independently distributed $\theta(z)$ estimated matrix still will be roughly the same:
\begin{multline*}
	\mathbb{E}\mcm(L, N) = \dfrac{1}{2^N} \left( \overline{e^{i\frac{\omega}{c}n_x\frac{L}{N}}} + \overline{e^{i\frac{\omega}{c}n_y\frac{L}{N}}}\right)^N \mathds{1} = \dfrac{1}{2^N} \left( \Phi_{n_x}\left(\frac{\omega}{c}\frac{L}{N}\right) + \Phi_{n_y}\left(\frac{\omega}{c}\frac{L}{N}\right)\right)^N \mathds{1}= \\
	= \left( 1 + i\dfrac{\omega}{c}\dfrac{\bar{n}_x+\bar{n}_y}{2}\dfrac{L}{N} + o\left(\dfrac{L}{N}\right)\right)^N  \mathds{1} \approx e^{i\frac{\omega L}{c}\frac{\bar{n}_x + \bar{n}_y}{2}}\mathds{1}
\end{multline*}
here $\Phi_{n_{x,y}}(t)$ - characteristic functions of $n_{x,y}$ distributions.

However mean Jones matrix is mostly out of interest. We really need estimation of so called $R$-parameter, which is by definition:
\begin{equation}
	\label{R}
	R = \max (R_x, R_y),\qquad
R_{x,y} = \dfrac{\int\limits_0^T\left|E_{x,y}(\tau)\right|^2d\tau}{\int\limits_0^T\left|E_x(\tau)\right|^2 + \left|E_y(\tau)\right|^2d\tau}
\end{equation}

We, in turn, have defined Jones matrix in frequency domain, so it will be convenient to rewrite~\eqref{R} with respect to Parseval's identity as:
\begin{equation*}
R_{x,y} = \frac{\sum\limits_{n=-\infty}^{+\infty} \left|E_{x,y}(\omega_n)\right|^2}{\sum\limits_{n=-\infty}^{+\infty} \left|E_x(\omega_n)\right|^2 + \left|E_y(\omega_n)\right|^2}
\end{equation*}

With respect to unitarity of $\mcm$:
\begin{equation*}
	R_{x,y} = \frac{\sum\limits_{n=-\infty}^{+\infty}\sum\limits_{\alpha,\beta} \mcm_{(x,y),\alpha}E^0_\alpha(\omega_n) E^{0*}_\beta(\omega_n) \mcm^\dagger_{\beta,(x,y)}}{\sum\limits_{n=-\infty}^{+\infty} \left|E_x^0(\omega_n)\right|^2 + \left|E_y^0(\omega_n)\right|^2}
\end{equation*}

In order to derive formula for $\mathbb{E}R_{x,y}$ we will need second-order correlators for $\mcm$: $\mathbb{E}\mcm^\dagger_{\alpha\beta}\mcm_{\sigma\rho}$. Remember, that $\mcm$ is~\eqref{M}, and $B(z,\Delta z)$ in case of no simplifications is:
\begin{equation*}
B(z, \Delta z) =  \begin{pmatrix}
	e^{i\frac{\omega}{c}n_x(z)\Delta z} \cos^2\theta(z) + e^{i\frac{\omega}{c}n_y(z) \Delta z} \sin^2\theta(z) & (e^{i\frac{\omega}{c}n_x(z)\Delta z} - e^{i\frac{\omega}{c}n_y(z) \Delta z}) \sin\theta(z)\cos\theta(z) \\
	(e^{i\frac{\omega}{c}n_x(z)\Delta z} - e^{i\frac{\omega}{c}n_y(z) \Delta z}) \sin\theta(z)\cos\theta(z) & e^{i\frac{\omega}{c}n_x(z)\Delta z} \sin^2\theta(z) + e^{i\frac{\omega}{c}n_y(z) \Delta z} \cos^2\theta(z) 
\end{pmatrix}
\end{equation*}

After averaging with assumption of $\theta(z), n_{x,y}(z)$ independence we will obtain for correlators:
\begin{multline*}
\mathbb{E}\mcm^\dagger_{\alpha\beta}\mcm_{\sigma\rho} = \mathbb{E}
\sum\limits_{\alpha_j,\beta_j}\delta_{\alpha\alpha_1}\delta_{\beta\alpha_{N+1}}\delta_{\sigma\beta_1}\delta_{\rho\beta_{N+1}}\prod\limits_{j=1}^{N} B^\dagger_{\alpha_j\alpha_{j+1}}\left( \dfrac{L}{N}(j-1), \dfrac{L}{N} \right)B_{\beta_{j+1}\beta_j}\left( \dfrac{L}{N}(j-1), \dfrac{L}{N} \right) = \\ =
\sum\limits_{\alpha_j,\beta_j}\delta_{\alpha\alpha_1}\delta_{\beta\alpha_{N+1}}\delta_{\sigma\beta_1}\delta_{\rho\beta_{N+1}}\prod\limits_{j=1}^{N}\mathbb{E} \left[B^\dagger_{\alpha_j\alpha_{j+1}}\left( \dfrac{L}{N}(j-1), \dfrac{L}{N} \right)B^T_{\beta_j\beta_{j+1}}\left( \dfrac{L}{N}(j-1), \dfrac{L}{N} \right)\right]
\end{multline*}

For estimations in right side we can explicitly write:
\begin{multline}
	\label{BEst}
\mathbb{E} \left[B^\dagger_{\alpha_j\alpha_{j+1}}\left( \dfrac{L}{N}(j-1), \dfrac{L}{N} \right)B^T_{\beta_j\beta_{j+1}}\left( \dfrac{L}{N}(j-1), \dfrac{L}{N} \right)\right] = \\ = \dfrac{1}{8} 
\scalemath{0.7}{ \begin{pmatrix}
	\begin{pmatrix}
		6 + \Phi_{\Delta n}\left( \frac{\omega}{c}\frac{L}{N} \right) + \Phi_{\Delta n}\left( -\frac{\omega}{c}\frac{L}{N} \right) & 0 \\
		0 & 2 + 3\Phi_{\Delta n}\left( \frac{\omega}{c}\frac{L}{N} \right) + 3\Phi_{\Delta n}\left( -\frac{\omega}{c}\frac{L}{N} \right)
	\end{pmatrix} & 
	\begin{pmatrix}
		0 & 2 - \Phi_{\Delta n}\left( \frac{\omega}{c}\frac{L}{N} \right) - \Phi_{\Delta n}\left( -\frac{\omega}{c}\frac{L}{N} \right) \\
		2 - \Phi_{\Delta n}\left( \frac{\omega}{c}\frac{L}{N} \right) - \Phi_{\Delta n}\left( -\frac{\omega}{c}\frac{L}{N} \right) & 0
	\end{pmatrix} \\
	\begin{pmatrix}
		0 & 2 - \Phi_{\Delta n}\left( \frac{\omega}{c}\frac{L}{N} \right) - \Phi_{\Delta n}\left( -\frac{\omega}{c}\frac{L}{N} \right) \\
		2 - \Phi_{\Delta n}\left( \frac{\omega}{c}\frac{L}{N} \right) - \Phi_{\Delta n}\left( -\frac{\omega}{c}\frac{L}{N} \right) & 0
	\end{pmatrix} & 	
	\begin{pmatrix}
		2 + 3\Phi_{\Delta n}\left( \frac{\omega}{c}\frac{L}{N} \right) + 3\Phi_{\Delta n}\left( -\frac{\omega}{c}\frac{L}{N} \right) & 0 \\
		0 & 6 + \Phi_{\Delta n}\left( \frac{\omega}{c}\frac{L}{N} \right) + \Phi_{\Delta n}\left( -\frac{\omega}{c}\frac{L}{N} \right)
	\end{pmatrix}
\end{pmatrix} }
\end{multline}
where $\Phi_{\Delta n}(t) = \mathbb{E}e^{it(n_x - n_y)} = \Phi_{n_x}(t)\cdot\Phi_{n_y}(-t)$ is characteristic function of $\Delta n = n_x - n_y$

Now we need somehow compute tensor contraction of similar tensors. For this purpose we need next formula. Given tensor
$$
B_{\alpha\beta\sigma\rho} = \begin{pmatrix}
	\begin{pmatrix}
		a & 0 \\
		0 & b
	\end{pmatrix} & 
	\begin{pmatrix}
		0 & c \\
		c & 0
	\end{pmatrix} \\
	\begin{pmatrix}
		0 & c \\
		c & 0
	\end{pmatrix} &
	\begin{pmatrix}
		b & 0 \\
		0 & a
	\end{pmatrix}
\end{pmatrix}
$$
we may analytically show:


\begin{multline*}
	\mathcal{P}(B,N)_{\alpha\beta\sigma\rho} = \sum\limits_{\alpha_j,\beta_j}\delta_{\alpha\alpha_1}\delta_{\beta\alpha_{N+1}}\delta_{\sigma\beta_1}\delta_{\rho\beta_{N+1}}\prod\limits_{j=1}^{N} B_{\alpha_j\alpha_{j+1}\beta_j\beta_{j+1}} = \\ =
\scalemath{0.7}{\begin{pmatrix}
	\begin{pmatrix}
		\sum\limits_{k=0}^{\lfloor \frac{N}{2} \rfloor} \binom{N}{2k}a^{N-2k}c^{2k} & 0 \\
		0 & \sum\limits_{k=0}^{\lfloor \frac{N}{2} \rfloor} \binom{N}{2k}b^{N-2k}c^{2k}
	\end{pmatrix} & 
	\begin{pmatrix}
		0 & \sum\limits_{k=0}^{\lfloor \frac{N-1}{2} \rfloor} \binom{N}{2k+1}a^{N-2k-1}c^{2k+1} \\
		\sum\limits_{k=0}^{\lfloor \frac{N-1}{2} \rfloor} \binom{N}{2k+1}b^{N-2k-1}c^{2k+1} & 0
	\end{pmatrix} \\
	\begin{pmatrix}
		0 & \sum\limits_{k=0}^{\lfloor \frac{N-1}{2} \rfloor} \binom{N}{2k+1}b^{N-2k-1}c^{2k+1} \\
		\sum\limits_{k=0}^{\lfloor \frac{N-1}{2} \rfloor} \binom{N}{2k+1}a^{N-2k-1}c^{2k+1} & 0
	\end{pmatrix} &
	\begin{pmatrix}
		\sum\limits_{k=0}^{\lfloor \frac{N}{2} \rfloor} \binom{N}{2k}b^{N-2k}c^{2k} & 0 \\
		0 & \sum\limits_{k=0}^{\lfloor \frac{N}{2} \rfloor} \binom{N}{2k}a^{N-2k}c^{2k}
	\end{pmatrix}
\end{pmatrix}	} = \\ =
\dfrac{1}{2} \begin{pmatrix}
	\begin{pmatrix}
		(a+c)^N+(a-c)^N & 0 \\
		0 & (b+c)^N+(b-c)^N
	\end{pmatrix} & 
	\begin{pmatrix}
		0 & (a+c)^N-(a-c)^N \\
		(b+c)^N-(b-c)^N & 0
	\end{pmatrix} \\
	\begin{pmatrix}
		0 & (b+c)^N-(b-c)^N \\
		(a+c)^N-(a-c)^N & 0
	\end{pmatrix} &
	\begin{pmatrix}
		(b+c)^N+(b-c)^N & 0 \\
		0 & (a+c)^N+(a-c)^N
	\end{pmatrix}
\end{pmatrix}
\end{multline*}
// TODO: show it) //

Substituting $a,b,c$ from~\eqref{BEst} we finally obtain:
\begin{equation*}
\mathbb{E}\mcm^\dagger_{\alpha\beta}\mcm_{\sigma\rho} = 
%\scalemath{0.4}{ \begin{pmatrix}
%	\begin{pmatrix}
%		1 +\left(\frac{1}{2}+\frac{\Phi_{\Delta n}\left( \frac{\omega}{c}\frac{L}{N} \right) + \Phi_{\Delta n}\left(- \frac{\omega}{c}\frac{L}{N} \right)}{4}\right)^N & 0 \\
%		0 & \left(\frac{1}{2}+\frac{\Phi_{\Delta n}\left( \frac{\omega}{c}\frac{L}{N} \right) + \Phi_{\Delta n}\left(- \frac{\omega}{c}\frac{L}{N} \right)}{4}\right)^N + \left(\frac{\Phi_{\Delta n}\left( \frac{\omega}{c}\frac{L}{N} \right) + \Phi_{\Delta n}\left(- \frac{\omega}{c}\frac{L}{N} \right)}{2}\right)^N
%	\end{pmatrix} & 
%	\begin{pmatrix}
%		0 & 1-\left(\frac{1}{2}+\frac{\Phi_{\Delta n}\left( \frac{\omega}{c}\frac{L}{N} \right) + \Phi_{\Delta n}\left(- \frac{\omega}{c}\frac{L}{N} \right)}{4}\right)^N \\
%		\left(\frac{1}{2}+\frac{\Phi_{\Delta n}\left( \frac{\omega}{c}\frac{L}{N} \right) + \Phi_{\Delta n}\left(- \frac{\omega}{c}\frac{L}{N} \right)}{4}\right)^N- \left(\frac{\Phi_{\Delta n}\left( \frac{\omega}{c}\frac{L}{N} \right) + \Phi_{\Delta n}\left(- \frac{\omega}{c}\frac{L}{N} \right)}{2}\right)^N & 0
%	\end{pmatrix} \\
%	\begin{pmatrix}
%		0 & \left(\frac{1}{2}+\frac{\Phi_{\Delta n}\left( \frac{\omega}{c}\frac{L}{N} \right) + \Phi_{\Delta n}\left(- \frac{\omega}{c}\frac{L}{N} \right)}{4}\right)^N- \left(\frac{\Phi_{\Delta n}\left( \frac{\omega}{c}\frac{L}{N} \right) + \Phi_{\Delta n}\left(- \frac{\omega}{c}\frac{L}{N} \right)}{2}\right)^N \\
%		1 -\left(\frac{1}{2}+\frac{\Phi_{\Delta n}\left( \frac{\omega}{c}\frac{L}{N} \right) + \Phi_{\Delta n}\left(- \frac{\omega}{c}\frac{L}{N} \right)}{4}\right)^N & 0
%	\end{pmatrix} &
%	\begin{pmatrix}
%		\left(\frac{1}{2}+\frac{\Phi_{\Delta n}\left( \frac{\omega}{c}\frac{L}{N} \right) + \Phi_{\Delta n}\left(- \frac{\omega}{c}\frac{L}{N} \right)}{4}\right)^N+ \left(\frac{\Phi_{\Delta n}\left( \frac{\omega}{c}\frac{L}{N} \right) + \Phi_{\Delta n}\left(- \frac{\omega}{c}\frac{L}{N} \right)}{2}\right)^N & 0 \\
%		0 & 1 +\left(\frac{1}{2}+\frac{\Phi_{\Delta n}\left( \frac{\omega}{c}\frac{L}{N} \right) + \Phi_{\Delta n}\left(- \frac{\omega}{c}\frac{L}{N} \right)}{4}\right)^N
%	\end{pmatrix}
%\end{pmatrix}
%} = \\ =
\begin{pmatrix}
	\begin{pmatrix}
		\mathfrak{A} & 0 \\
		0 & \mathfrak{B}
	\end{pmatrix} & 
	\begin{pmatrix}
		0 & \mathfrak{C} \\
	\mathfrak{D} & 0
	\end{pmatrix} \\
	\begin{pmatrix}
		0 & \mathfrak{D} \\
		\mathfrak{C} & 0
	\end{pmatrix} &
	\begin{pmatrix}
		\mathfrak{B} & 0 \\
		0 & \mathfrak{A}
	\end{pmatrix}
\end{pmatrix}
\end{equation*}
where
\begin{align*}
	\mathfrak{A} &= \frac{1}{2} +\frac{1}{2}\left(\frac{1}{2}+\frac{\Phi_{\Delta n}\left( \frac{\omega}{c}\frac{L}{N} \right) + \Phi_{\Delta n}\left(- \frac{\omega}{c}\frac{L}{N} \right)}{4}\right)^N \\
	\mathfrak{B} &= \frac{1}{2}\left(\frac{1}{2}+\frac{\Phi_{\Delta n}\left( \frac{\omega}{c}\frac{L}{N} \right) + \Phi_{\Delta n}\left(- \frac{\omega}{c}\frac{L}{N} \right)}{4}\right)^N+ \frac{1}{2}\left(\frac{\Phi_{\Delta n}\left( \frac{\omega}{c}\frac{L}{N} \right) + \Phi_{\Delta n}\left(- \frac{\omega}{c}\frac{L}{N} \right)}{2}\right)^N \\
	\mathfrak{C} &= \frac{1}{2} -\frac{1}{2}\left(\frac{1}{2}+\frac{\Phi_{\Delta n}\left( \frac{\omega}{c}\frac{L}{N} \right) + \Phi_{\Delta n}\left(- \frac{\omega}{c}\frac{L}{N} \right)}{4}\right)^N \\
	\mathfrak{D} &= \frac{1}{2}\left(\frac{1}{2}+\frac{\Phi_{\Delta n}\left( \frac{\omega}{c}\frac{L}{N} \right) + \Phi_{\Delta n}\left(- \frac{\omega}{c}\frac{L}{N} \right)}{4}\right)^N- \frac{1}{2}\left(\frac{\Phi_{\Delta n}\left( \frac{\omega}{c}\frac{L}{N} \right) + \Phi_{\Delta n}\left(- \frac{\omega}{c}\frac{L}{N} \right)}{2}\right)^N
\end{align*}

We will exploit the following property of characteristic function:
\begin{equation*}
\Phi_{\Delta n}\left(t\right) = 1 + it\mathbb{E}\left[\Delta n\right] - \frac{t^2}{2}\mathbb{E}\left[\Delta n^2\right] + \underline{o}\left(t^2\right)
\end{equation*}

so, assuming $\frac{L}{N} \ll 1$ we may rewrite:
\begin{align*}
	\mathfrak{A} &= \frac{1}{2} + \frac{1}{2}\exp\left\{ -\left(\frac{\omega L}{2c}\right)^2\frac{\mathbb{E}\left[\Delta n^2\right]}{N} + \underline{o}\left(\frac{1}{N}\right) \right\} = 1 -\frac{1}{2}\left(\frac{\omega L}{2c}\right)^2\frac{\mathbb{E}\left[\Delta n^2\right]}{N} + \underline{o}\left(\frac{1}{N}\right) \\
	\mathfrak{B} &= \frac{1}{2}\exp\left\{ -\left(\frac{\omega L}{2c}\right)^2\frac{\mathbb{E}\left[\Delta n^2\right]}{N} + \underline{o}\left(\frac{1}{N}\right) \right\} + \frac{1}{2}\exp\left\{-\frac{1}{2} \left(\frac{\omega L}{c}\right)^2\frac{\mathbb{E}\left[\Delta n^2\right]}{N} + \underline{o}\left(\frac{1}{N}\right) \right\} \\
	\mathfrak{C} &= \frac{1}{2} - \frac{1}{2} \exp\left\{ -\left(\frac{\omega L}{2c}\right)^2\frac{\mathbb{E}\left[\Delta n^2\right]}{N} + \underline{o}\left(\frac{1}{N}\right) \right\} = \frac{1}{2}\left(\frac{\omega L}{2c}\right)^2\frac{\mathbb{E}\left[\Delta n^2\right]}{N} + \underline{o}\left(\frac{1}{N}\right) \\
	\mathfrak{D} &= \frac{1}{2}\exp\left\{ -\left(\frac{\omega L}{2c}\right)^2\frac{\mathbb{E}\left[\Delta n^2\right]}{N} + \underline{o}\left(\frac{1}{N}\right) \right\} - \frac{1}{2}\exp\left\{-\frac{1}{2} \left(\frac{\omega L}{c}\right)^2\frac{\mathbb{E}\left[\Delta n^2\right]}{N} + \underline{o}\left(\frac{1}{N}\right) \right\}
\end{align*}
and finally, for $R_x$:
\begin{multline}
	\label{REst}
	R_{x,y} = \frac{\sum\limits_{n=-\infty}^{+\infty} \mathfrak{A}\left|E^0_{x,y}(\omega_n)\right|^2 + \mathfrak{C}\left|E^0_{y,x}(\omega_n)\right|^2}{\sum\limits_{n=-\infty}^{+\infty} \left|E_x^0(\omega_n)\right|^2 + \left|E_y^0(\omega_n)\right|^2} = \\ =
	R^0_{x,y} - \frac{1}{2}\left(\frac{L}{c}\right)^2\frac{\mathbb{E}\left[\Delta n^2\right]}{N} \frac{\sum\limits_{n=-\infty}^{+\infty} \omega_n^2\left(\left|E^0_{x,y}(\omega_n)\right|^2 - \left|E^0_{y,x}(\omega_n)\right|^2\right)}{\sum\limits_{n=-\infty}^{+\infty} \left|E_x^0(\omega_n)\right|^2 + \left|E_y^0(\omega_n)\right|^2} + \underline{o}\left(\frac{1}{N}\right)
\end{multline}

Formula~\eqref{REst} shows, that in limit $\Delta L \rightarrow 0$\,($N\rightarrow \infty$) there will be no change in $R$-coefficient due to optical fiber.

The conclusion we can make from these properties is that approximation of optical fiber with one having completely independently distributed parameters have no practical interest and in general is wrong approach. To caption effects of polarization corruption one need to model optical fiber with care of correlation effects between different points or strictly limit $\Delta L$ from below by correlation length. However in this case given characteristic functions of $n_{x,y}$ distributions one may explicitly write analytical expression for $R$.

\section{Numerical solution}
In order to obtain SM optical fiber Jones matrix one need to solve~\eqref{main2},~\eqref{true} for given distribution of $\theta(z)$ and $n_{x,y}(z)$ along $z$-axis. However, in case of no losses it can be simplified to~\eqref{mainvec}. For this purpose we propose to use well-known ODE solving methods from Runge-Kutta family. One main advantage of this methods is that we can preserve matrix or vector notation. In other words there is no need to rewrite~\eqref{main2},\eqref{mainvec} as a system of equations on each element of $\mcm$, which keeps visibility. In vector case one may also explicitly include preservation of determinant in solver.

Some words need to be said about modeling of a set of fibers. As we saw in~\ref{problem} the usage of uniform and independent distribution of $\theta(z)$ leads to useless bunch of calculations. So it will be better to find some other distributions, which fulfills more physical properties, such as continuity or even smoothness.
	
\end{document}