% \documentclass[12pt,a4paper]{article}
\documentclass[12pt, notitlepage]{report}
\usepackage[utf8]{inputenc}
\usepackage[T1]{fontenc}


\usepackage[left=1in, right=1in, top=1in, bottom=1in]{geometry}

\usepackage{titling}
%\usepackage{lipsum}

\usepackage{amsmath}
\usepackage{amsfonts}
\usepackage{amssymb}
\usepackage{graphicx}
\usepackage{float}
%\usepackage[margin=2cm]{geometry}
\usepackage{caption}
\usepackage{dsfont}

\pretitle{\begin{center}\Huge\bfseries}
	\posttitle{\par\end{center}\vskip 0.5em}
\preauthor{\begin{center}\Large\ttfamily}
	\postauthor{\end{center}}
\predate{\par\large\centering}
\postdate{\par}

\title{Stability of polarizations in VCSEL}
\author{Evgeniy Dedkov}
\date{\today}
\begin{document}
	
	\newcommand\scalemath[2]{\scalebox{#1}{\mbox{\ensuremath{\displaystyle #2}}}}
	\newcommand{\F}{\boldsymbol{\mathfrak{F}}}
	\newcommand{\n}{\boldsymbol{\mathfrak{n}}}
	\newcommand{\N}{\boldsymbol{\mathcal{N}}}
	\newcommand{\Q}{\mathcal{Q}}
	\newcommand{\gp}{\gamma_{\parallel}}
	
	\maketitle
	\thispagestyle{empty}
	
	\begin{abstract}
		This material derives regions of stability of different polarizations in VCSEL.
	\end{abstract}
	
	\section{Basic equations}
	We will use the following notation for rate equations of VCSEL:
	\begin{gather}
		\dot{E_+} = \frac{1}{2\tau_{ph}}(1+i\alpha)\left[G_L - 1+ \frac{d}{N_{th} - N_{tr}}\right] E_+ - (\gamma_a + i\gamma_p) E_- \\
		\dot{E_-} = \frac{1}{2\tau_{ph}}(1+i\alpha)\left[G_L - 1- \frac{d}{N_{th} - N_{tr}}\right] E_- - (\gamma_a + i\gamma_p) E_+ \\
		\dot{N} = \frac{I}{e} - \frac{N}{\tau_e} - \frac{G_L}{\tau_{ph}}(|E_+|^2+|E_-|^2) - \frac{1}{\tau_{ph}}\frac{d}{N_{th} - N_{tr}}(|E_+|^2 - |E_-|^2)\\
		\dot{d} = -\frac{1}{\tau_d} d -  \frac{G_L}{\tau_{ph}}(|E_+|^2-|E_-|^2) - \frac{1}{\tau_{ph}}\frac{d}{N_{th} - N_{tr}}(|E_+|^2+|E_-|^2)
	\end{gather}
	where $G_L = \frac{N - N_{tr}}{N_{th} - N_{tr}}$
	For simplicity lets introduce:
	\begin{gather}
		\F_\pm = \sqrt{\frac{\tau_{e}}{\tau_{ph}(N_{th} - N_{tr)}}}E_\pm \\
		\n = \frac{d}{N_{th} - N_{tr}} \\
		\N = G_L
	\end{gather}
	So resulting rate equations will be (Roman - 4.32):
	\begin{gather}
		\label{main_rate1}
		\dot{\F_+} = \kappa(1+i\alpha)\left[\N + \n - 1\right] \F_+ - (\gamma_a + i\gamma_p) \F_- \\
		\dot{\F_-} = \kappa(1+i\alpha)\left[\N - \n - 1\right] \F_- - (\gamma_a + i\gamma_p) \F_+ \\
		\dot{\N} = \gp\left(\mu-\N\right) - \gp\N(|\F_+|^2+|\F_-|^2) - \gp\n(|\F_+|^2 - |\F_-|^2) \\
		\label{main_rate4}
		\dot{\n} = -\gamma_d \n - \gp\N(|\F_+|^2-|\F_-|^2) - \gp\n(|\F_+|^2+|\F_-|^2)
	\end{gather}
	
	\section{Stationary solutions}
	To obtain stationary solutions for elliptically polarized light we need to substitute
	\begin{gather*}
		\F_\pm = \sqrt{\Q_\pm} e^{i(\omega t \pm \psi)}\\
		\N = \N_s\\
		\n = \n_s
	\end{gather*}
	in \eqref{main_rate1}--\eqref{main_rate4}. We obtain:
	\begin{gather}
		\label{stat1}
		i\omega = \kappa(1+i\alpha)\left[\N_s+ \n_s - 1\right] - (\gamma_a + i\gamma_p) \sqrt{\frac{\Q_-}{\Q_+}} e^{- 2i\psi} \\
		\label{stat2}
		i\omega = \kappa(1+i\alpha)\left[\N_s - \n_s - 1\right] - (\gamma_a + i\gamma_p) \sqrt{\frac{\Q_+}{\Q_-}} e^{2i\psi} \\
		\label{stat3}
		0 = \gp\left(\mu-\N_s\right) - \gp\N_s(\Q_+ + \Q_-) - \gp\n_s(\Q_+ - \Q_-) \\
		\label{stat4}
		0 = -\gamma_d \n_s - \gp\N_s(\Q_+ - \Q_-) - \gp\n_s(\Q_+ + \Q_-)
	\end{gather}

Lets solve~\eqref{stat3},~\eqref{stat4} for $\N_s$, $n_s$. Here and further we will omit s-indices of stationary values for convenience and readability.
\begin{gather}
	\label{eq_N}
	\N = A\mu\cdot \left[\gp (\Q_+ + \Q_-) + \gamma_d\right] \\
	\label{eq_n}
	\n = A\mu\gp(\Q_- - \Q_+) \\
	A^{-1} = {\gamma_d + \left(\Q_+ +\Q_-\right)\left(\gp + \gamma_d\right) + 4\gp\Q_+\Q_-}
\end{gather}

	First two equations may be rewritten as:
	\begin{gather}
		\label{statir1}
		0 = \kappa \left[\N+ \n - 1\right] - \gamma_a \sqrt{\frac{\Q_-}{\Q_+}} \cos 2\psi - \gamma_p \sqrt{\frac{\Q_-}{\Q_+}} \sin 2\psi \\
		\label{statir2}
		\omega = \alpha\kappa \left[\N+ \n - 1\right] + \gamma_a \sqrt{\frac{\Q_-}{\Q_+}} \sin 2\psi - \gamma_p \sqrt{\frac{\Q_-}{\Q_+}} \cos 2\psi \\
		\label{statir3}
		0 = \kappa \left[\N - \n - 1\right] - \gamma_a \sqrt{\frac{\Q_+}{\Q_-}} \cos 2\psi + \gamma_p \sqrt{\frac{\Q_+}{\Q_-}} \sin 2\psi \\
		\label{statir4}
		\omega = \alpha\kappa \left[\N- \n - 1\right] - \gamma_a \sqrt{\frac{\Q_+}{\Q_-}} \sin 2\psi - \gamma_p \sqrt{\frac{\Q_+}{\Q_-}} \cos 2\psi
	\end{gather}

From here we can observe, that if there exist solution $\psi_0, \Q_\pm^0, \omega^0, \N^0, \n^0$, there will exist one more: 
\begin{gather*}
	\psi^1 = -\psi^0 \\
	\Q_\pm^1 = \Q_\mp^0 \\
	\omega^1 = \omega^0 \\
	\N^1 = \N^0 \\
	\n^1 = -\n^0
\end{gather*}

Taking a look at combinations: \eqref{statir2}-$\alpha$\eqref{statir1} and \eqref{statir4}-$\alpha$\eqref{statir3} one may find representations for $\omega$:
\begin{gather}
	\omega = \sqrt{\frac{\Q_-}{\Q_+}}\left[\left(\alpha\gamma_a-\gamma_p\right)\cos2\psi+\left(\gamma_a+\alpha\gamma_p\right)\sin2\psi\right] \\
	\omega = \sqrt{\frac{\Q_+}{\Q_-}}\left[\left(\alpha\gamma_a-\gamma_p\right)\cos2\psi - \left(\gamma_a+\alpha\gamma_p\right)\sin2\psi\right]
\end{gather} 

After-all, we are left with 3 equations, which are~\eqref{statir1},~\eqref{statir3} and \eqref{statir2} minus \eqref{statir4}:
\begin{gather}
	\kappa\left(A\mu (2\gp\Q_- + \gamma_d) - 1\right) - \sqrt{\frac{\Q_-}{\Q_+}}\left(\gamma_a\cos2\psi + \gamma_p\sin2\psi\right) = 0 \\
	\kappa\left(A\mu (2\gp\Q_+ + \gamma_d) - 1\right) - \sqrt{\frac{\Q_+}{\Q_-}}\left(\gamma_a\cos2\psi - \gamma_p\sin2\psi\right) = 0 \\
	2\alpha\kappa A\mu\gp(\Q_- - \Q_+)  + \left( \sqrt{\frac{\Q_+}{\Q_-}} - \sqrt{\frac{\Q_-}{\Q_+}} \right)\gamma_p\cos2\psi + \left( \sqrt{\frac{\Q_+}{\Q_-}} + \sqrt{\frac{\Q_-}{\Q_+}} \right)\gamma_a\sin2\psi = 0
\end{gather}

All we need is to solve them numerically, which should be performed with appropriate trigonometric substitution.

There are three following reasonable choices:
\begin{enumerate}
	\item $$
	\cos2\psi \rightarrow \pm\sqrt{1-a^2},\qquad \sin2\psi \rightarrow a
	$$
	which is used by Roman.
	\item $$
	\cos2\psi \rightarrow a,\qquad \sin2\psi \rightarrow \pm\sqrt{1-a^2}
	$$
	which is probably the best one, since there are no troubles with sign: we may choose any. However, there are some troubles in wolfram with this one substitution.
	\item
	 $$
	\cos2\psi \rightarrow \frac{1-a^2}{1+a^2},\qquad \sin2\psi \rightarrow \frac{2a}{1+a^2}
	$$ 
	which is universal trigonometric substitution.
\end{enumerate} 

\section{Linearization and stability}
Since we obtain stationary solutions we should analyze them in terms of stability. For this purpose we propose to linearize initial system and apply Lyapunov conditions. 
\begin{gather}
	\F_\pm = \left(\sqrt{\Q_\pm}+a_\pm\right)e^{i(\omega t \pm \psi)} \\
	\N = \N_s + \Delta \\
	\n = \n_s + \delta
\end{gather}

After leaving only first order terms, we end with:
\newcommand{\bv}{\boldsymbol{v}}
\newcommand{\bM}{\boldsymbol{M}}
\begin{equation}
	\dot{\bv} = \bM\bv
\end{equation}
where
\begin{equation}
	\bv = \begin{pmatrix}
		\Re\left\{a_+\right\} \\
		\Im\left\{a_+\right\} \\
		\Re\left\{a_-\right\} \\
		\Im\left\{a_-\right\} \\
		\Delta \\
		\delta
	\end{pmatrix}
\end{equation}

\begin{equation}
	\bM = \begin{pmatrix}
		m_{11} & m_{12} & \cdots & m_{16} \\
		m_{21} & m_{22} & \cdots &m_{26}\\
		\vdots & \vdots & \ddots & \vdots\\
		m_{61} & m_{62} & \cdots&  m_{66} 
	\end{pmatrix}
\end{equation}
and for $m_{ij}$:
\begin{align}
	m_{11} &= m_{22} = \kappa \left(\N_s + \n_s - 1\right)\\
	m_{12} &= -m_{21} = -\alpha\kappa\left(\N_s+\n_s-1\right) + \omega\\
	m_{33} &= m_{44} = \kappa \left(\N_s - \n_s - 1\right) \\
	m_{34} &= -m_{43} = - \alpha\kappa \left(\N_s - \n_s - 1\right) + \omega \\
	m_{13} &= m_{24} = -\gamma_a\cos2\psi - \gamma_p\sin2\psi \\
	m_{14} &= -m_{23} =  \gamma_p\cos2\psi - \gamma_a\sin2\psi \\
	m_{31} &= m_{42} =  -\gamma_a\cos2\psi + \gamma_p\sin2\psi \\
	m_{41} &= -m_{32} =  -\gamma_p\cos2\psi - \gamma_a\sin2\psi \\
	m_{55} &= -\gp\left(\Q_+ + \Q_- \right) - \gp\\
	m_{66} &= - \gp\left(\Q_+ + \Q_-\right) -\gamma_d \\
	m_{56} &= m_{65} = \gp\left(\Q_- - \Q_+\right)	\\
	m_{15} &= m_{16} = \kappa\sqrt{\Q_+}\\
	m_{25} &= m_{26} = \alpha\kappa\sqrt{\Q_+}\\
	m_{35} &= -m_{36} = \kappa\sqrt{\Q_-}\\
	m_{45} &= -m_{46} = \alpha\kappa\sqrt{\Q_-}\\
	m_{51} &= m_{61} = -2\gp\sqrt{\Q_+}\left(\N_s + \n_s\right) \\
	m_{53} &= -m_{63} = -2\gp\sqrt{\Q_-}\left(\N_s - \n_s\right) \\
\end{align}

here we may use equalities deriven from~\eqref{statir1}--\eqref{statir4} and \eqref{stat3}, \eqref{stat4} to express coefficients in different form:
\begin{gather}
	\kappa \left(\N_s + \n_s - 1\right) = \sqrt{\frac{\Q_-}{\Q_+}} \left( \gamma_a  \cos 2\psi + \gamma_p  \sin 2\psi \right) \\
	-\alpha\kappa \left(\N_s+ \n_s - 1\right) + \omega  = -\sqrt{\frac{\Q_-}{\Q_+}} \left(\gamma_p  \cos 2\psi - \gamma_a  \sin 2\psi \right) \\
	\kappa \left(\N_s - \n_s - 1\right) = \sqrt{\frac{\Q_+}{\Q_-}} \left(\gamma_a  \cos 2\psi - \gamma_p \sin 2\psi\right) \\
	-\alpha\kappa \left(\N_s- \n_s - 1\right) + \omega  = -\sqrt{\frac{\Q_+}{\Q_-}} \left(\gamma_p \cos 2\psi + \gamma_a  \sin 2\psi\right)
\end{gather}
so, it's now clear, that:
\begin{gather*}
	m_{11} = -\sqrt{\frac{\Q_-}{\Q_+}}m_{13} \\
	m_{12} = -\sqrt{\frac{\Q_-}{\Q_+}}m_{14} \\
	m_{33} = -\sqrt{\frac{\Q_+}{\Q_-}}m_{31} \\
	m_{34} = \sqrt{\frac{\Q_+}{\Q_-}}m_{41} \\
\end{gather*}

It is better to use defenctions through $\Q_\pm$ and $\psi$ in our approach, since they are obtained through direct numerical solution without any redundant steps, so are more precise.

One more we should note about $\bM$ is that:
\begin{equation}
	\det\bM = 0
\end{equation}
since
\begin{equation}
	\bM \begin{pmatrix}
		0 & \sqrt{Q_+} &  0 & \sqrt{Q_-} & 0 & 0
	\end{pmatrix}^T = 0
\end{equation}

This proof could be found from notion, that we can choose arbitrary phase, so for every $\varphi$ $\F_\pm = \sqrt{\Q_\pm}\exp\left\{ i(\omega t \pm \psi + \varphi) \right\}$ will be also the solution for stationary problem, so we may expect appropriate $a_\pm$ to be the proof of zero-determinant. However:
\begin{equation}
	\bM \begin{pmatrix}
		-\sqrt{Q_+}(1-\cos\varphi) \\
		\sqrt{Q_+}\sin\varphi \\
		-\sqrt{Q_-}(1-\cos\varphi) \\
		\sqrt{Q_-}\sin\varphi \\
		0 \\
		0
	\end{pmatrix} = \begin{pmatrix}
	0 \\
	0 \\
	0 \\
	0 \\
	4\gp (\mu - \N_s)\sin^2\frac{\varphi}{2} \\
	-4\gp \gamma_d\n_s\sin^2\frac{\varphi}{2}
\end{pmatrix}
\end{equation}
which is quadratic on $\varphi \ll 1$. So, performing partial differentiation on both sides we obtain:
\begin{equation}
	\bM \begin{pmatrix}
		0 & \sqrt{Q_+} &  0 & \sqrt{Q_-} & 0 & 0
	\end{pmatrix}^T = 0
\end{equation}
which proves that determinant of $\bM$ is zero.

Next we need to evaluate characteristic polynomial:
\begin{equation}
	f(s) = a_0 s^6 + a_1 s^5 + a_2 s^4 + a_3 s^3 + a_4 s ^2 + a_5 s
\end{equation}
and perform analysis via Hurwitz criterion. One may argue on its inapplicability since we have non-positive zero coefficient $a_7$. But, the zero root we obtain cannot lead to anything, but initial phase change. So small perturbations in it have no effect on the system at all. So, we may just ignore it and perform analysis with the rest of characteristic polynomial.

Now, lets assume all coefficients are positive (we will check necessary Stodola criterion). So, Hurwitz criterion now can be rewritten:
\begin{gather}
	\Delta_2 = a_1 a_2 - a_0 a_3 > 0 \\
	\Delta_4 = -a_1^2 a_4^2 + a_1 (a_2 a_3 a_4 - a_2^2 a_5 + 2 a_0 a_4 a_5) - 
	a_0 (a_3^2 a_4 - a_2 a_3 a_5 + a_0 a_5^2) > 0
\end{gather}

%\begin{gather}
%	\Delta_2 = a_1 a_2 - a_0 a_3 > 0 \\
%	\Delta_3 = a_3\Delta_2 - a_1^2a_4 + a_5a_0a_1 > 0 \\
%	\Delta_4 = a_4\Delta_3 + a_0a_5(a_2a_3-a_0a_5) -a_1(a_2^2a_5-a_0a_4a_5-a_6\Delta_2) > 0\\
%	\Delta_5 = a_5\Delta_4 + a_6(a_0a_3^3-a_1a_3(a_2a_3+2a_0a_5)+a_1^2(a_3a_4+a_2a_5)-a_1^3a_6) > 0
%\end{gather}

So, algorithm will be the follow:
\begin{enumerate}
	\item Obtain stationary values for given parameters
	\item Construct matrix of linearized system
	\item Obtain coefficients of characteristic polynomial
	\item Check necessary Stodola criterion
	\item Check Hurwitz criterion
\end{enumerate}

\subsection{Linear polarizations and stability}
It can be shown, that in case of linear polarization, which means $\Q_+=\Q_-=\Q$ equations~\eqref{stat3}--\eqref{stat4}, \eqref{statir1}--\eqref{statir4} become much simpler, and one may obtain analytical solution for x-polarization:
\begin{gather}
	\Q^{(x)} = \frac{1}{2} \left(\frac{\kappa  \mu }{\gamma_a + \kappa } - 1\right)\\
	\omega^{(x)} = \alpha \gamma_a - \gamma_p\\
	\psi^{(x)} = 0\\
	\N_s^{(x)} = \frac{\mu}{1+2\Q} = 1 + \frac{\gamma_a}{\kappa}\\
	\n_s^{(x)} = 0
\end{gather}
and for y-polarization:
\begin{gather}
	\Q^{(y)} = \frac{1}{2} \left(\frac{\kappa  \mu }{\kappa - \gamma_a } - 1\right)\\
	\omega^{(y)} = -\alpha \gamma_a + \gamma_p\\
	\psi^{(y)} = \frac{\pi}{2}\\
	\N_s^{(y)} = \frac{\mu}{1+2\Q} = 1 - \frac{\gamma_a}{\kappa}\\
	\n_s^{(y)} = 0
\end{gather}

Note, that there can be no other variants of linear polarization. 

Then one should perform stability analysis. For this purpose wi will rewrite linearized equations in terms of new variables:
\begin{gather}
	R_+ = \Re a_+ + \Re a_- \\
	I_+ = \Im a_+ + \Im a_- \\
	R_- = \Re a_+ - \Re a_- \\
	I_- = \Im a_+ - \Im a_-
\end{gather}
then, we obtain two independent systems:
\begin{gather}
	\left\{\begin{align}
		&\dot{R}_+ = 2\kappa\sqrt{\Q}\Delta \\
		&\dot{I}_+ = 2\alpha\kappa\sqrt{\Q}\Delta \\
		&\dot{\Delta} = -2\gp\sqrt{\Q}\N_s R_+ - \gp(1+2\Q)\Delta
	\end{align}\right.\\
	\left\{\begin{align}
		&\dot{R}_- = \pm2\gamma_a R_- \mp 2\gamma_p I_- + 2\kappa\sqrt{\Q}\delta \\
		&\dot{I}_- = \pm2\gamma_p R_- \pm 2\gamma_a I_- + 2\alpha\kappa\sqrt{\Q}\delta \\
		&\dot{\delta} = -2\gp\sqrt{\Q}\N_s R_- - (2\gp\Q + \gamma_d)\delta
	\end{align}\right.
\end{gather}
for x- and y- polarization respectively.

For the first system we have characteristic polynomial coefficients:
\begin{gather}
	a_0 = 1\\
	a_1 = \gp(1+2\Q)\\
	a_2 = 4\gp\kappa\Q\N_s \\
	a_3 = 0
\end{gather}
which gives only trivial condition $\N_s > 0$.

For the second system:
\begin{gather}
	b_0 = 1\\
	b_1 = 2\gp\Q +\gamma_d \mp 4\gamma_a \\
	b_2 = 4 \left(\gamma_a^2 + \gamma_p^2 \mp \gamma_a\gamma_d \mp 2 \gp\gamma_a\Q\right)  + 4\gp\kappa\Q \N_s\\
	b_3 = 4 \left(\gamma_a^2+\gamma_p^2\right) (\gamma_d + 2 \gp\Q) \mp 8 \gp\kappa\Q (\gamma_a + \alpha  \gamma_p)\N_s
\end{gather}
is much more complicated. We need to satisfy the following conditions:
\begin{gather}
	b_1 > 0,\\
	b_3 > 0,\\
	b_1b_2 - b_3 > 0,
\end{gather}
which after substitution becomes (for x-polarization):
\begin{gather}
	\mu > \frac{(\gamma_a+\kappa ) (\gp +4 \gamma_a - \gamma_d)}{\gp  \kappa },\\
	\mu < \frac{(\gamma_a+\kappa ) \left(\gp  \gamma_a (\alpha  \gamma_p+\kappa )+\gamma_p (\alpha  \gp  \kappa -\gp  \gamma_p+\gamma_d \gamma_p)+\gamma_a^2 \gamma_d\right)}{\gp  \kappa  \left(\alpha  \gamma_p (\gamma_a+\kappa )+\gamma_a \kappa -\gamma_p^2\right)},\\
	\begin{aligned}
	\mu > \frac{\gamma_a+\kappa }{\kappa } + \frac{\gamma_a+\kappa }{2 \gp ^2 \kappa ^2 (\gamma_a-\kappa )}\left[
	\gp  \kappa  \left(\gamma_a (2 \alpha  \gamma_p-3 \gamma_d-2 \kappa )+\kappa  (2 \alpha  \gamma_p+\gamma_d)+6 \gamma_a^2\right)
	\right] + \\
	+ \sqrt{\gp ^2 \kappa ^2 \left(\gamma_a^2 \left(4 \left(\alpha ^2-8\right) \gamma_p^2+12 \alpha  \gamma_p (2 \gamma_a-\gamma_d)+(\gamma_d-2 \gamma_a)^2\right)+2 \gamma_a \kappa  \left(4 \left(\alpha ^2+4\right) \gamma_p^2+4 \alpha  \gamma_p (2 \gamma_a-\gamma_d)+(\gamma_d-2 \gamma_a)^2\right)+\kappa ^2 (2 \alpha  \gamma_p-2 \gamma_a+\gamma_d)^2\right)}
	\end{aligned}
\end{gather}
\end{document}